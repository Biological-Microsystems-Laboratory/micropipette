	% Template for PLoS
% Version 3.1 February 2015
%
% To compile to pdf, run:
% latex plos.template
% bibtex plos.template
% latex plos.template
% latex plos.template
% dvipdf plos.template
%
% % % % % % % % % % % % % % % % % % % % % %
%
% -- IMPORTANT NOTE
%
% This template contains comments intended 
% to minimize problems and delays during our production 
% process. Please follow the template instructions
% whenever possible.
%
% % % % % % % % % % % % % % % % % % % % % % % 
%
% Once your paper is accepted for publication, 
% PLEASE REMOVE ALL TRACKED CHANGES in this file and leave only
% the final text of your manuscript.
%
% There are no restrictions on package use within the LaTeX files except that 
% no packages listed in the template may be deleted.
%
% Please do not include colors or graphics in the text.
%
% Please do not create a heading level below \subsection. For 3rd level headings, use \paragraph{}.
%
% % % % % % % % % % % % % % % % % % % % % % %
%
% -- FIGURES AND TABLES
%
% Please include tables/figure captions directly after the paragraph where they are first cited in the text.
%
% DO NOT INCLUDE GRAPHICS IN YOUR MANUSCRIPT
% - Figures should be uploaded separately from your manuscript file. 
% - Figures generated using LaTeX should be extracted and removed from the PDF before submission. 
% - Figures containing multiple panels/subfigures must be combined into one image file before submission.
% For figure citations, please use "Fig." instead of "Figure".
% See http://www.plosone.org/static/figureGuidelines for PLOS figure guidelines.
%
% Tables should be cell-based and may not contain:
% - tabs/spacing/line breaks within cells to alter layout or alignment
% - vertically-merged cells (no tabular environments within tabular environments, do not use \multirow)
% - colors, shading, or graphic objects
% See http://www.plosone.org/static/figureGuidelines#tables for table guidelines.
%
% For tables that exceed the width of the text column, use the adjustwidth environment as illustrated in the example table in text below.
%
% % % % % % % % % % % % % % % % % % % % % % % %
%
% -- EQUATIONS, MATH SYMBOLS, SUBSCRIPTS, AND SUPERSCRIPTS
%
% IMPORTANT
% Below are a few tips to help format your equations and other special characters according to our specifications. For more tips to help reduce the possibility of formatting errors during conversion, please see our LaTeX guidelines at http://www.plosone.org/static/latexGuidelines
%
% Please be sure to include all portions of an equation in the math environment.
%
% Do not include text that is not math in the math environment. For example, CO2 will be CO\textsubscript{2}.
%
% Please add line breaks to long display equations when possible in order to fit size of the column. 
%
% For inline equations, please do not include punctuation (commas, etc) within the math environment unless this is part of the equation.
%
% % % % % % % % % % % % % % % % % % % % % % % % 
%
% Please contact latex@plos.org with any questions.
%
% % % % % % % % % % % % % % % % % % % % % % % %

\documentclass[10pt,letterpaper]{article}
\usepackage[top=0.85in,left=2.75in,footskip=0.75in]{geometry}

% Use adjustwidth environment to exceed column width (see example table in text)
\usepackage{changepage}

% Use Unicode characters when possible
\usepackage[utf8]{inputenc}

% textcomp package and marvosym package for additional characters
\usepackage{textcomp,marvosym}

% fixltx2e package for \textsubscript
\usepackage{fixltx2e}

% amsmath and amssymb packages, useful for mathematical formulas and symbols
\usepackage{amsmath,amssymb}

% cite package, to clean up citations in the main text. Do not remove.
\usepackage{cite}

% Use nameref to cite supporting information files (see Supporting Information section for more info)
\usepackage{nameref,hyperref}

% line numbers
\usepackage[right]{lineno}

% ligatures disabled
\usepackage{microtype}
\DisableLigatures[f]{encoding = *, family = * }

% rotating package for sideways tables
\usepackage{rotating}

\usepackage{multirow}

% Remove comment for double spacing
%\usepackage{setspace} 
%\doublespacing

% Text layout
\raggedright
\setlength{\parindent}{0.5cm}
\textwidth 5.25in 
\textheight 8.75in

% Bold the 'Figure #' in the caption and separate it from the title/caption with a period
% Captions will be left justified
\usepackage[aboveskip=1pt,labelfont=bf,labelsep=period,justification=raggedright,singlelinecheck=off]{caption}

% Use the PLoS provided BiBTeX style
\bibliographystyle{plos2015}

% Remove brackets from numbering in List of References
\makeatletter
\renewcommand{\@biblabel}[1]{\quad#1.}
\makeatother

% Leave date blank
\date{}

% Header and Footer with logo
\usepackage{lastpage,fancyhdr,graphicx}
\usepackage{epstopdf}
\pagestyle{myheadings}
\pagestyle{fancy}
\fancyhf{}
\lhead{\includegraphics[width=2.0in]{PLOS-submission.eps}}
\rfoot{\thepage/\pageref{LastPage}}
\renewcommand{\footrule}{\hrule height 2pt \vspace{2mm}}
\fancyheadoffset[L]{2.25in}
\fancyfootoffset[L]{2.25in}
\lfoot{\sf PLOS}

%% Include all macros below

\newcommand{\lorem}{{\bf LOREM}}
\newcommand{\ipsum}{{\bf IPSUM}}

%% END MACROS SECTION


\begin{document}
\vspace*{0.35in}

% Title must be 250 characters or less.
% Please capitalize all terms in the title except conjunctions, prepositions, and articles.
\begin{flushleft}
{\Large
\textbf\newline{3D-Printable Micropipette}
}
\newline
% Insert author names, affiliations and corresponding author email (do not include titles, positions, or degrees).
\\
Martin D. Brennan\textsuperscript{1},
Fahad F. Bokhari\textsuperscript{1},
David T. Eddington\textsuperscript{1,*},
%Name4 Surname\textsuperscript{2,\ddag},
%Name5 Surname\textsuperscript{2,\ddag},
%Name6 Surname\textsuperscript{2},
%Name7 Surname\textsuperscript{3,*},
%with the Lorem Ipsum Consortium\textsuperscript{\textpilcrow}
\\
\bigskip
\bf{1} Department of Bioengineering, University of Illinois at Chicago, Chicago, Illinois, USA
%\\
%\bf{2} Affiliation Dept/Program/Center, Institution Name, City, State, Country
%\\
%\bf{3} Affiliation Dept/Program/Center, Institution Name, City, State, Country
%\\
\bigskip

% Insert additional author notes using the symbols described below. Insert symbol callouts after author names as necessary.
% 
% Remove or comment out the author notes below if they aren't used.
%
% Primary Equal Contribution Note
%\Yinyang These authors contributed equally to this work.

% Additional Equal Contribution Note
% Also use this double-dagger symbol for special authorship notes, such as senior authorship.
%\ddag These authors also contributed equally to this work.

% Current address notes
%\textcurrency a Insert current address of first author with an address update
% \textcurrency b Insert current address of second author with an address update
% \textcurrency c Insert current address of third author with an address update

% Deceased author note
%\dag Deceased

% Group/Consortium Author Note
%\textpilcrow Membership list can be found in the Acknowledgments section.

% Use the asterisk to denote corresponding authorship and provide email address in note below.
* dte@uic.edu

\end{flushleft}
% Please keep the abstract below 300 words
\section*{Abstract}
Open source development of lab equipment is being facilitated, in part, by growing availability of 3D printing which allows the production of simple lab tools.
We developed 3D-printable parts that, along with a disposable syringe, form a micropipette.
Once assembled the pipette requires no calibration or validation with a scale.
Our printed micropipette is assessed in comparison to a commercial pipette demonstrating comparable performance in accuracy and precision and approaches ISO standard.

\linenumbers

\section*{Introduction}
The open source development model, initially applied to software, is thriving in the development of open source scientific equipment due in part to increasing access of 3D printing \cite{Baden2015,Pearce2014}.
Additive manufacturing methods have existed for decades although the recent availability of inexpensive desktop printers\cite{MakerbotIndustries2016,Reprap2015} have made it feasible for consumers to design and print prototypes and even functional parts, and consumer goods\cite{Fullerton2014}.
Proliferation of free CAD software[OpenSCAD, Blender, SketchUp] and design sharing sites\cite{MakerbotIndustries2016,NationalInstitutesofHealth2015,grabCAD,GitHubInc.2016} have also supported the growth and popularity of open designed parts and projects.
Open design 3D-printable lab equipment is and attractive idea because, like open source software, it allows free access to technology that is otherwise inaccessible due to proprietary and/or financial barriers. 
Open design tools create opportunity for scientists and educational programs in remote or resource limited areas to participate with inexpensive and easy to make tools\cite{Baden2015}. 
Open source development also enables the development of custom solutions to meet unique applications not met by commercial products that are shared freely and are user modifiable\cite{Fullerton2014,Pearce2012}.
Some advanced, noteworthy, open-source scientific equipment include a PCR device\cite{ChaiBiotechnologiesInc2015}, and a two-photon microscope\cite{Rosenegger2014} although simple tools are often more impactful as they can serve a wider community.
Some simple and clever printable parts that have emerged are ones that give a new function to a ubiquitous existing device, such as drill bit attachment designed to hold centrifuge tubes, allowing a drill to be used as a centrifuge\cite{Garvey2009}.
Although this may make a rather crude centrifuge it may be an adequate method which only costs pennies compared to commercially available centrifuges.
Other examples of open-design research tools that utilize 3D printed parts include optics equipment\cite{Zhang2013}, microscopes\cite{Baden2014a,Walus2014}, syringe pumps\cite{Wijnen2014}, and reactionware\cite{Symes2012} to name a few.

One example of a everyday scientific tool that provides opportunity for an open design solution is the micropipette.
Micropipettes are and indispensable tool used routinely in lab tasks and can easily cost \$1000 USD for a set.
Often a lab will require several sets each for a dedicated task.
Some pipettes are even re-calibrated for use with liquids of different properties.
A open design pipette that can be made for cheap can cut costs for labs as well as allow a option for educational settings.

Air displacement pipettes use a piston operating principal to draw liquid into the pipette\cite{ISO2002}.
In a typical commercial pipette the piston is made to be gas-tight with a gasketed plunger inside a smooth barrel.
Consumer grade fused deposition modeling (FDM) printers are unable to build a smooth surface do to the formation of ridges that occur as each layer is deposited.
The ridges formed by FDM make it impractical to form a gas-tight seal between moving parts, even with a gasket.
Existing printable open-design micropipettes get around this limitation by stretching a membrane over one end of a printed tube, which when pressed causes the displacement.
The displacement membrane can be made from any elastic material such as a latex glove.
There are a few open design printable micropipettes including a popular one which, in addition to the printed parts, uses parts scavenged from a retractable pen \cite{Baden2014}.
Because there is no built-in feature such as a readout for the user to set the displacement to a desired volume, this design requires the user to validate the volumes dispensed with a high precision scale.
Without verification with a scale the volumes dispensed can only be estimated based on calculations of the deflection of the membrane.

We submit a new design whose major strength is the ability to adjust to any volume according to built in scale.
Our open design 3D-printable micropipette works by actuating a disposable syringe to a user adjustable set-point.
This allows the user to set the pipette to a volume by reading the graduations on the syringe barrel.
The scale marks on the syringe can be used to adjust the dispensed volume accurately without calibrating with a scale.
Our pipette also offers a simplified assembly requiring no glue, tape or permanent connections.
%In addition no open-design pipettes have be rigorously validated.

%range of 226 ul, one tip size per pipette



%\begin{equation}\label{eq:schemeP} 
%D_{coll} = \frac{D_f+\frac{[S]^2}{K_D S_T} D_S} {1+\frac{[S]^2}{K_D S_T}}, 
%D_{sm} = \frac{D_f+ \frac{[S]}{K_D} D_S}{1+\frac{[S]}{K_D}},
%\end{equation}

% You may title this section "Methods" or "Models". 
% "Models" is not a valid title for PLoS ONE authors. However, PLoS ONE
% authors may use "Analysis" 
\section*{Materials and Methods}

Our printed pipette is designed to actuate a 1 mL or 3 mL syringe to a user set displacement.
The core of the design is two printed parts, the body and the plunger which are able to be printed on a consumer grade FDM printer.
A 1 mL or 3 mL BD syringe twists to lock in the body part and is held into place by the syringe flanges.
The 30-300 uL configuration uses a 1 mL pipette and the 100-1000 uL configuration uses a 3 mL pipette.
The plunger part slides freely in the body part and actuates the syringe by pushing the thumb press. 
The pipette is spring loaded towards a set point which is adjustable by a set-screw. 
When the plunger is depressed the system locks when it reaches the latched position, where it is ready to draw in fluid.
The plunger is held in place with a latching button design which is released with a lever drawing in fluid.
The displacement is equal to the distance between the set position and the latched position.
The pipette can also be pressed past the latched position to 'blow-out' the transfered fluid completely from the pipette tip.   
Additional materials required for assembly include a spring, a nut and bolt and washers.
Attempts to make a printable luer lock adapter for tips was abandoned as the surface of printed parts is too rough to make an air tight seal with the luer or pipette tip.
Instead a combination of a barbed luer adapter and elastic tubing is used to attach the pipette tips.
Our printed pipette mimics commercial pipettes's design, function, and user operation, making it intuitive to use.
Conveniently this design can also reach to the bottom of a 15 mL conical tube allowing tasks such as aspirating supernatant fluid from a cell pellet.

Our pipette uses the air-displacement design, where a pocket of air is used to draw liquid into the pipette.
As air is an compressible fluid, this pocket of air grows due to the weight of the liquid pulling on it.
Due to this effect the graduations on the syringe are not accurate, as they are designed for measuring liquid within the syringe.
At larger volumes this effect is more pronounced resulting in the volume measured being greater then the amount of liquid pulled into the syringe.
We remedied this by creating a new scale to account for the expansion.
The scale is printed on a transparency sheet and is pasted on the syringe for accurate measurements. 

\subsection*{Assembly}

A small amount of parrifin wax is applyed to the screw to prevent slop from causeing the set point to drift after each acuation.
The nut is sunk in to the hex inset in the body part.
The bolt is threaded in from the top of the body into the nut.
Springs and washers are threaded onto the plunger of the 1 mL syringe for the 30-300 uL configuration.
Springs are placed inside the 3 mL syringe for the 100-1000 mL configuration.
The plunger part is inserted in the body and the syringe assembly is pushed in and locked from the syringe flanges to the body part to complete assembly.

\begin{table}[!ht]
\centering
\caption{Parts and cost for the 30-300 uL pipette.}
\label{table1}
\begin{tabular}{|l|l|l|l|}
	\hline
	Part         & Unit Price & Source         & Part number \\ \hline
	Filament     & \$1.63     & Makerbot       & NA          \\ \hline
	1 mL Syringe & \$0.15     & BD Biosciences & 309628      \\ \hline
	Bolt         & \$0.12     & McMaster-Carr  & 91287A026   \\ \hline
	Nut          & \$0.01     & McMaster-Carr  & 90591A121   \\ \hline
	Spring (2)   & \$4.14     & McMaster-Carr  & 94125K542   \\ \hline
	Washers (2)  & \$0.16     & McMaster-Carr  & 90107A012   \\ \hline
	total        & \$6.21     &                &             \\ \hline
\end{tabular}
\end{table}

\begin{table}[!ht]
\centering
\caption{Parts and cost for the 100-1000 uL pipette.}
\label{table2}
\begin{tabular}{|l|l|l|l|}
	\hline
	Part         & Unit Price & Source         & Part number \\ \hline
	Filament     & \$1.63     & Makerbot       & NA          \\ \hline
	3 mL Syringe & \$0.73     & BD Biosciences & 309657      \\ \hline
	Bolt         & \$0.12     & McMaster-Carr  & 91287A026   \\ \hline
	Nut          & \$0.01     & McMaster-Carr  & 90591A121   \\ \hline
	Spring (2)   & \$4.14     & McMaster-Carr  & 94125K542   \\ \hline
	total        & \$6.63     &                &             \\ \hline
\end{tabular}
\end{table}

\begin{figure}
\caption{
{\bf CAD renderings of printable parts.} Two parts are printed: The plunger part (A) slides inside the body part (B).
}
\label{fig1}
\end{figure}

\begin{figure}
\caption{
{\bf CAD renderings of assembled pipette and function.} The pipetted actuates the syringe to three positions. {\bf(i) The set position.} The position of the screw determines the total displacement. The pipette is spring loaded to return to this position {\bf (ii) The latched position.} When the plunger is pressed the pipette locks at this position. The tip is then placed in a liquid and the button pressed to release the pipette back to the set position, drawing in liquid. {\bf (iii) The blow-out position.} The fluid is transfered by pressing the plunger past the latched position to blow-out all the liquid.  
}
\label{fig1}
\end{figure}

\begin{figure}
\caption{
{\bf Photos of pipette.}  The pipette is composed of two printed parts, a 1 mL syringe and some additional hardware.  
}
\label{fig3}
% Ideally: Photo showing assembled pipette as well as parts
\end{figure}

\subsection*{Validation}
The printed pipette’s accuracy and precision was characterized and compared to a commercial pipette as well as ISO 8655.
The printed pipette was adjusted to the the target volume by eye from the syringe graduations, and the resulting volume was measured and repeated 5 times to account for transfer variability.
Data was taken for printed pipettes with existing graduations as well as with the adjusted scale.
The commerical pipette was adjusted to the desired volume and 5 transfers were taken.

% For figure citations, please use "Fig." instead of "Figure".
%Nulla mi mi, Fig.~\ref{fig1} venenatis sed ipsum varius, volutpat euismod diam. Proin rutrum vel massa non gravida. Quisque tempor sem et dignissim rutrum. Lorem ipsum dolor sit amet, consectetur adipiscing elit. Morbi at justo vitae nulla elementum commodo eu id massa. In vitae diam ac augue semper tincidunt eu ut eros. Fusce fringilla erat porttitor lectus cursus, \nameref{S1_Video} vel sagittis arcu lobortis. Aliquam in enim semper, aliquam massa id, cursus neque. Praesent faucibus semper libero.

%\begin{enumerate}
%\item{react}
%\item{diffuse free particles}
%\item{increment time by dt and go to 1}
%\end{enumerate}

% Results and Discussion can be combined.
\section*{Results and Discussion}

Initially validation was preformed with the original graduations printed on the syringes.
Due to the expansion of air the built in graduations are were not accurate, especially at larger volumes.
Using our redesigned scale volumes were accurate.
Our printed pipette meets ISO standards for accuracy and precision. (Table~\ref{table1 ,table2}).

This pipette improves on existing open design pipettes in several ways. 
It requires only two printed parts, a syringe, and some hardware.
It is able to reach into a 15 mL conical tube which is a routine requirement for cell culture.
No permanent connections using tape or glue are required for assembly allowing worn out or broken part to be replaced easily.
Assembly does not require a membrane that may wear or stretch and require tedious replacement.
No validation or calibration is required.
The pipette can be assembled and used to accurately pipette without first validating with a scale.

The only major limitation of this design compared to the briopipette is the option for a pipette tip ejector, although incorporation of a similar design in our pipette would not leave room for it to reach into a 15 mL conical tube. 
Our pipette requires a syringe and some additional hardware where briopipette's additional parts can be obtained virtually anywhere.
The briopipette also allows step-wise shifts in displacement allowing the user to quickly change to a relative volume


% Please add the following required packages to your document preamble:
% \usepackage{multirow}
\begin{table}[!ht]
\begin{adjustwidth}{-2.25in}{0in} % Comment out/remove adjustwidth environment if table fits in text column.
\centering
\caption{ISO 8655 for 100-1000 uL comparing a commercial pipette with our printed pipette used with existing 3 mL syringe scale and an adjusted scale}
\label{table3}
\begin{tabular}{lllllll}
        &                       & Mean   & Systematic Error & \% Sys. err. & Random Error & \% Rand. err. \\
1000 uL & ISO 8655, 100-1000 uL & 1000    & 8.00             & 0.80         & 3.00         & 0.30          \\
        & Commercial Pipette    & 1002.98 & 2.98             & 0.30         & 1.72         & 0.17          \\
        & Printed Pipette       & 949.29  & -50.71           & -5.07        & 0.60         & 0.06          \\
        & Printed Pipette Scale & 1003.57 & 3.57             & 0.36         & 0.89         & 0.09          \\
        &                       &         &                  &              &              &               \\
500 uL  & ISO 8655, 100-1000 uL & 500     & 8.00             & 1.60         & 3.00         & 0.60          \\
        & Commercial Pipette    & 503.67  & 3.67             & 0.73         & 0.49         & 0.10          \\
        & Printed Pipette       & 475.99  & -24.01           & -4.80        & 4.75         & 1.00          \\
        & Printed Pipette Scale & 503.62  & 3.62             & 0.72         & 1.64         & 0.33          \\
        &                       &         &                  &              &              &               \\
200 uL  & ISO 8655, 100-1000 uL & 200     & 8.00             & 4.00         & 3.00         & 1.50          \\
        & Commercial Pipette    & 204.61  & 4.61             & 2.30         & 0.15         & 0.07          \\
        & Printed Pipette       & 186.55  & -13.45           & -6.72        & 1.31         & 0.70          \\
        & Printed Pipette Scale & 201.87  & 1.87             & 0.94         & 1.47         & 0.73          \\
        &                       &         &                  &              &              &               \\
100 uL  & ISO 8655, 100-1000 uL & 100     & 8.00             & 8.00         & 3.00         & 3.00          \\
        & Commercial Pipette    & 104.29  & 4.29             & 4.29         & 1.65         & 1.58          \\
        & Printed Pipette       & 94.02   & -5.98            & -5.98        & 4.81         & 5.12          \\
        & Printed Pipette Scale & 101.00  & 1.00             & 1.00         & 1.05         & 1.04         
\end{tabular}
%\begin{flushleft} Table notes Phasellus venenatis, tortor nec vestibulum mattis, massa tortor interdum felis, nec pellentesque metus tortor nec nisl. Ut ornare mauris tellus, vel dapibus arcu suscipit sed.
%\end{flushleft}
\end{adjustwidth}
\end{table}

\begin{table}[!ht]
\begin{adjustwidth}{-2.25in}{0in} % Comment out/remove adjustwidth environment if table fits in text column.
\centering
\caption{ISO 8655 for 30-300 uL comparing a commercial pipette with our printed pipette used with existing 3 mL syringe scale and an adjusted scale}
\label{table4}
\begin{tabular}{lllllll}
       &                       & Mean   & Systematic Error & \% Sys. err. & Random Error & \% Rand. err. \\
300 uL & ISO 8655, 30-300 uL   & 300    & 4.00             & 1.33         & 1.50         & 0.50          \\
       & Commercial Pipette    & 301.19 & 1.19             & 0.40         & 0.53         & 0.18          \\
       & Printed Pipette       & 286.91 & -13.09           & -4.36        & 0.42         & 0.15          \\
       & Printed Pipette Scale & 299.11 & -0.89            & -0.30        & 0.48         & 0.16          \\
       &                       &        &                  &              &              &               \\
200 uL & ISO 8655, 30-300 uL   & 200    & 4                & 2            & 1.5          & 0.75          \\
       & Commercial Pipette    & 200.06 & 0.06             & 0.03         & 0.46         & 0.23          \\
       & Printed Pipette       & 193.40 & -6.60            & -3.30        & 2.86         & 1.48          \\
       & Printed Pipette Scale & 200.57 & 0.57             & 0.28         & 0.86         & 0.43          \\
       &                       &        &                  &              &              &               \\
50 uL  & ISO 8655, 30-300 uL   & 50     & 4                & 8            & 1.5          & 3             \\
       & Commercial Pipette    & 49.02  & -0.98            & -1.96        & 0.10         & 0.20          \\
       & Printed Pipette       & 49.62  & -0.38            & -0.76        & 1.26         & 2.53          \\
       & Printed Pipette Scale & 48.73  & -1.27            & -2.54        & 1.11         & 2.27          \\
       &                       &        &                  &              &              &               \\
30 uL  & ISO 8655, 30-300 uL   & 30     & 4                & 13.3         & 1.5          & 5             \\
       & Commercial Pipette    & 29.08  & -0.92            & -3.06        & 0.09         & 0.31          \\
       & Printed Pipette       & 29.22  & -0.78            & -2.59        & 0.31         & 1.07          \\
       & Printed Pipette Scale & 27.78  & -2.22            & -7.41        & 1.37         & 4.93          \\
       &                       &        &                  &              &              &               \\
20 uL* & ISO 8655, 30-300 uL   & 20     & 4                & 20           & 1.5          & 7.5           \\
       & Commercial Pipette    & NA     & NA               & NA           & NA           & NA            \\
       & Printed Pipette       & 18.70  & -1.30            & -6.48        & 0.38         & 2.01          \\
       & Printed Pipette Scale & 17.94  & -2.06            & -10.29       & 1.87         & 10.42         \\
       &                       &        &                  &              &              &               \\
10 uL* & ISO 8655, 30-300 uL   & 10     & 4                & 40           & 1.5          & 15            \\
       & Commercial Pipette    & NA     & NA               & NA           & NA           & NA            \\
       & Printed Pipette       & 11.95  & 1.95             & 19.52        & 0.73         & 6.08          \\
       & Printed Pipette Scale & 7.64   & -2.36            & -23.64       & 0.38         & 4.92         
\end{tabular}
\begin{flushleft} * The 20 uL and 10 uL volumes are out of the range in this case but we wanted to demonstrate that the pipette is capable of even smaller volumes.
\end{flushleft}
\end{adjustwidth}
\end{table}

%\begin{table}[!ht]
%\begin{adjustwidth}{-2.25in}{0in} % Comment out/remove adjustwidth environment if table fits in text column.
%\caption{
%{\bf Table caption Nulla mi mi, venenatis sed ipsum varius, volutpat euismod diam.}}
%\begin{tabular}{|l|l|l|l|l|l|l|l|}
%\hline
%\multicolumn{4}{|l|}{\bf Heading1} & \multicolumn{4}{|l|}{\bf Heading2}\\ \hline
%$cell1 row1$ & cell2 row 1 & cell3 row 1 & cell4 row 1 & cell5 row 1 & cell6 row 1 & cell7 row 1 & cell8 row 1\\ \hline
%$cell1 row2$ & cell2 row 2 & cell3 row 2 & cell4 row 2 & cell5 row 2 & cell6 row 2 & cell7 row 2 & cell8 row 2\\ \hline
%$cell1 row3$ & cell2 row 3 & cell3 row 3 & cell4 row 3 & cell5 row 3 & cell6 row 3 & cell7 row 3 & cell8 row 3\\ \hline
%\end{tabular}
%\begin{flushleft} Table notes Phasellus venenatis, tortor nec vestibulum mattis, massa tortor interdum felis, nec pellentesque metus tortor nec nisl. Ut ornare mauris tellus, vel dapibus arcu suscipit sed.
%\end{flushleft}
%\label{table1}
%\end{adjustwidth}
%\end{table}

% Please do not create a heading level below \subsection. For 3rd level headings, use \paragraph{}. 

For more information, see \nameref{S1_Text}.

\section*{Supporting Information}

% Include only the SI item label in the subsection heading. Use the \nameref{label} command to cite SI items in the text.
%\subsection*{S1 Video}
%\label{S1_Video}
%{\bf Bold the first sentence.}  Maecenas convallis mauris sit amet sem ultrices gravida. Etiam eget sapien nibh. Sed ac ipsum eget enim egestas ullamcorper nec euismod ligula. Curabitur fringilla pulvinar lectus consectetur pellentesque.

%\subsection*{S1 Text}
%\label{S1_Text}
%{\bf Lorem Ipsum.} Maecenas convallis mauris sit amet sem ultrices gravida. Etiam eget sapien nibh. Sed ac ipsum eget enim egestas ullamcorper nec euismod ligula. Curabitur fringilla pulvinar lectus consectetur pellentesque.

%\subsection*{S1 Fig}
%\label{S1_Fig}
%{\bf Lorem Ipsum.} Maecenas convallis mauris sit amet sem ultrices gravida. Etiam eget sapien nibh. Sed ac ipsum eget enim egestas ullamcorper nec euismod ligula. Curabitur fringilla pulvinar lectus consectetur pellentesque.

%\subsection*{S2 Fig}
%\label{S2_Fig}
%{\bf Lorem Ipsum.} Maecenas convallis mauris sit amet sem ultrices gravida. Etiam eget sapien nibh. Sed ac ipsum eget enim egestas ullamcorper nec euismod ligula. Curabitur fringilla pulvinar lectus consectetur pellentesque.

%\subsection*{S1 Table}
%\label{S1_Table}
%{\bf Lorem Ipsum.} Maecenas convallis mauris sit amet sem ultrices gravida. Etiam eget sapien nibh. Sed ac ipsum eget enim egestas ullamcorper nec euismod ligula. Curabitur fringilla pulvinar lectus consectetur pellentesque.

%\section*{Acknowledgments}
%Cras egestas velit mauris, eu mollis turpis pellentesque sit amet. Interdum et malesuada fames ac ante ipsum primis in faucibus. Nam id pretium nisi. Sed ac quam id nisi malesuada congue. Sed interdum aliquet augue, at pellentesque quam rhoncus vitae.

\nolinenumbers

%\section*{References}
% Either type in your references using
% \begin{thebibliography}{}
% \bibitem{}
% Text
% \end{thebibliography}
%
% OR
%
% Compile your BiBTeX database using our plos2015.bst
% style file and paste the contents of your .bbl file
% here.
% 
\bibliographystyle{plos2015.bst}
\bibliography{references}



\end{document}

