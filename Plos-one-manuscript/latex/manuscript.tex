% Template for PLoS
% Version 3.1 February 2015
%
% To compile to pdf, run:
% latex plos.template
% bibtex plos.template
% latex plos.template
% latex plos.template
% dvipdf plos.template
%
% % % % % % % % % % % % % % % % % % % % % %
%
% -- IMPORTANT NOTE
%
% This template contains comments intended 
% to minimize problems and delays during our production 
% process. Please follow the template instructions
% whenever possible.
%
% % % % % % % % % % % % % % % % % % % % % % % 
%
% Once your paper is accepted for publication, 
% PLEASE REMOVE ALL TRACKED CHANGES in this file and leave only
% the final text of your manuscript.
%
% There are no restrictions on package use within the LaTeX files except that 
% no packages listed in the template may be deleted.
%
% Please do not include colors or graphics in the text.
%
% Please do not create a heading level below \subsection. For 3rd level headings, use \paragraph{}.
%
% % % % % % % % % % % % % % % % % % % % % % %
%
% -- FIGURES AND TABLES
%
% Please include tables/figure captions directly after the paragraph where they are first cited in the text.
%
% DO NOT INCLUDE GRAPHICS IN YOUR MANUSCRIPT
% - Figures should be uploaded separately from your manuscript file. 
% - Figures generated using LaTeX should be extracted and removed from the PDF before submission. 
% - Figures containing multiple panels/subfigures must be combined into one image file before submission.
% For figure citations, please use "Fig." instead of "Figure".
% See http://www.plosone.org/static/figureGuidelines for PLOS figure guidelines.
%
% Tables should be cell-based and may not contain:
% - tabs/spacing/line breaks within cells to alter layout or alignment
% - vertically-merged cells (no tabular environments within tabular environments, do not use \multirow)
% - colors, shading, or graphic objects
% See http://www.plosone.org/static/figureGuidelines#tables for table guidelines.
%
% For tables that exceed the width of the text column, use the adjustwidth environment as illustrated in the example table in text below.
%
% % % % % % % % % % % % % % % % % % % % % % % %
%
% -- EQUATIONS, MATH SYMBOLS, SUBSCRIPTS, AND SUPERSCRIPTS
%
% IMPORTANT
% Below are a few tips to help format your equations and other special characters according to our specifications. For more tips to help reduce the possibility of formatting errors during conversion, please see our LaTeX guidelines at http://www.plosone.org/static/latexGuidelines
%
% Please be sure to include all portions of an equation in the math environment.
%
% Do not include text that is not math in the math environment. For example, CO2 will be CO\textsubscript{2}.
%
% Please add line breaks to long display equations when possible in order to fit size of the column. 
%
% For inline equations, please do not include punctuation (commas, etc) within the math environment unless this is part of the equation.
%
% % % % % % % % % % % % % % % % % % % % % % % % 
%
% Please contact latex@plos.org with any questions.
%
% % % % % % % % % % % % % % % % % % % % % % % %

\documentclass[10pt,letterpaper]{article}
\usepackage[top=0.85in,left=2.75in,footskip=0.75in]{geometry}

% Use adjustwidth environment to exceed column width (see example table in text)
\usepackage{changepage}

% Use Unicode characters when possible
\usepackage[utf8]{inputenc}

% textcomp package and marvosym package for additional characters
\usepackage{textcomp,marvosym}

% fixltx2e package for \textsubscript
\usepackage{fixltx2e}

% amsmath and amssymb packages, useful for mathematical formulas and symbols
\usepackage{amsmath,amssymb}

% cite package, to clean up citations in the main text. Do not remove.
\usepackage{cite}

% Use nameref to cite supporting information files (see Supporting Information section for more info)
\usepackage{nameref,hyperref}

% line numbers
\usepackage[right]{lineno}

% ligatures disabled
\usepackage{microtype}
\DisableLigatures[f]{encoding = *, family = * }

% rotating package for sideways tables
\usepackage{rotating}

% Remove comment for double spacing
%\usepackage{setspace} 
%\doublespacing

% Text layout
\raggedright
\setlength{\parindent}{0.5cm}
\textwidth 5.25in 
\textheight 8.75in

% Bold the 'Figure #' in the caption and separate it from the title/caption with a period
% Captions will be left justified
\usepackage[aboveskip=1pt,labelfont=bf,labelsep=period,justification=raggedright,singlelinecheck=off]{caption}

% Use the PLoS provided BiBTeX style
\bibliographystyle{plos2015}

% Remove brackets from numbering in List of References
\makeatletter
\renewcommand{\@biblabel}[1]{\quad#1.}
\makeatother

% Leave date blank
\date{}

% Header and Footer with logo
\usepackage{lastpage,fancyhdr,graphicx}
\usepackage{epstopdf}
\pagestyle{myheadings}
\pagestyle{fancy}
\fancyhf{}
\lhead{\includegraphics[width=2.0in]{PLOS-submission.eps}}
\rfoot{\thepage/\pageref{LastPage}}
\renewcommand{\footrule}{\hrule height 2pt \vspace{2mm}}
\fancyheadoffset[L]{2.25in}
\fancyfootoffset[L]{2.25in}
\lfoot{\sf PLOS}

%% Include all macros below

\newcommand{\lorem}{{\bf LOREM}}
\newcommand{\ipsum}{{\bf IPSUM}}

%% END MACROS SECTION


\begin{document}
\vspace*{0.35in}

% Title must be 250 characters or less.
% Please capitalize all terms in the title except conjunctions, prepositions, and articles.
\begin{flushleft}
{\Large
\textbf\newline{3D-Printable Micropipette}
}
\newline
% Insert author names, affiliations and corresponding author email (do not include titles, positions, or degrees).
\\
Martin D. Brennan\textsuperscript{1},
Fahad F. Bokhari\textsuperscript{1},
David T. Eddington\textsuperscript{1,*},
%Name4 Surname\textsuperscript{2,\ddag},
%Name5 Surname\textsuperscript{2,\ddag},
%Name6 Surname\textsuperscript{2},
%Name7 Surname\textsuperscript{3,*},
%with the Lorem Ipsum Consortium\textsuperscript{\textpilcrow}
\\
\bigskip
\bf{1} Department of Bioengineering, University of Illinois at Chicago, Chicago, Illinois, USA
%\\
%\bf{2} Affiliation Dept/Program/Center, Institution Name, City, State, Country
%\\
%\bf{3} Affiliation Dept/Program/Center, Institution Name, City, State, Country
%\\
\bigskip

% Insert additional author notes using the symbols described below. Insert symbol callouts after author names as necessary.
% 
% Remove or comment out the author notes below if they aren't used.
%
% Primary Equal Contribution Note
%\Yinyang These authors contributed equally to this work.

% Additional Equal Contribution Note
% Also use this double-dagger symbol for special authorship notes, such as senior authorship.
%\ddag These authors also contributed equally to this work.

% Current address notes
%\textcurrency a Insert current address of first author with an address update
% \textcurrency b Insert current address of second author with an address update
% \textcurrency c Insert current address of third author with an address update

% Deceased author note
%\dag Deceased

% Group/Consortium Author Note
%\textpilcrow Membership list can be found in the Acknowledgments section.

% Use the asterisk to denote corresponding authorship and provide email address in note below.
* dte@uic.edu

\end{flushleft}
% Please keep the abstract below 300 words
\section*{Abstract}
This is the abstract.

\linenumbers

\section*{Introduction}
3D printing as a form of additive manufacturing has existed for decades although the recent availability of inexpensive desktop printers have made it feasible for consumers to design and print prototypes and even functional parts, and consumer goods\cite{Fullerton2014}.
The open source software movement and has spawned the additional areas of open source hardware, such as micro controller boards (Arduino) as well as 3D printers themselves (RepRap). 
Open source scientific instruments and laboratory equipment are being actively developed(openpcr.org, opensource lab book.) 
3D-printable lab equipment is and an attractive idea because of the simplicity of downloading and printing a functional object for only the price of the raw material\cite{Waldbaur2011,Zhang2013}. 
In addition, end users can customize the equipment to their application. 
Some of the more clever printable parts that have emerged are ones that give a new function to a ubiquitous existing device. 
For example a drill bit attachment was designed and printed to hold micro centrifuge tubes allowing a drill to be used for centrifuging samples. 
Although this may make a rather crude centrifuge it may be and adequate method that only costs pennies compared to commercially available centrifuges and could allow researchers in the developing world to better participate in science by reducing some financial barriers of entry. 
Printable micropipette designs can be found on 3D-part sharing websites (thingiverse). 
While existing designs may require only pen springs, a gasket made from a balloon, and tape, none allow adjustment to know volumes nor have any been rigorously validated. 
We present a pipette constructed from 3D printed parts that allows a 1 ml syringe to be actuated to an adjustable volume. 
The scale marks on the syringe can be used to adjust the dispensed volume accurately without calibrating with a scale.

%\begin{equation}\label{eq:schemeP} 
%D_{coll} = \frac{D_f+\frac{[S]^2}{K_D S_T} D_S} {1+\frac{[S]^2}{K_D S_T}}, 
%D_{sm} = \frac{D_f+ \frac{[S]}{K_D} D_S}{1+\frac{[S]}{K_D}},
%\end{equation}

% You may title this section "Methods" or "Models". 
% "Models" is not a valid title for PLoS ONE authors. However, PLoS ONE
% authors may use "Analysis" 
\section*{Materials and Methods}
\subsection*{Design}

The device is designed to actuate a pipette to a set distance. The pipette is plunger is actuated to three points, two fixed and one adjustable. 
The fixed points are the dispensed point, when the plunger is all the way down, and the reset point, when the plunger is ready to withdraw fluid. 
The adjustable point is the set point, which determines the volume of fluid that will be drawn into the pipette and depends on the position of the screw. 
The system is spring loaded towards the set point. 
A spring notch, in the plunger shaft part, and a groove, in the body part, hold the plunger at the reset position where it can be released by pressing in the notch. 
A 1 ml BD syringe twists to lock into the body part. 
An additional printed part adapts standard micropipette tips to the luer fitting of the pipette.
Our printed pipette mimics commercial pipettes is design, function, and user operation, and is intuitive to use.
3D parts were designed in Blender and SolidWorks and printed by Fineline Prototyping via selective laser sintering. 
Additional materials include a spring, a nut and bolt and two washers.

\begin{figure}
\caption{
{\bf CAD rendering of printable parts.}  Three parts are printed: the body, plunger shaft, and luer-lock adapter for pipette tips.
}
\label{fig1}
\end{figure}

\begin{figure}
\caption{
{\bf Photos of pipette.}  The pipette is composed of three printed parts, a 1 mL syringe and some additional hardware.  
}
\label{fig2}
\end{figure}

\subsection*{Validation}
The pipette was validated and compared to a commercial pipette. 
The pipette was tested to see if the set point would drift between during use for example small, incremental movements of the screw or other parts causing the pipetted volume to drift. 
To test the pipette the same volume was pipetted times without moving or re-adjusting the set point. 
The printed pipette’s accuracy and precision was characterized and compared to a commercial pipette. 
The printed pipette and a series of commercial pipettes were tested at volumes of 20 $\mu$L 50 $\mu$L and 200 $\mu$L. 
The printed pipette was adjusted to the the target volume by eye from the syringe graduations, and the resulting volume was measured and repeated 5 times to account for variability.
Measurements were taken from the commercial pipette in a similar way.
To test user friendliness and feasibility for using the pipette without verifying with a scale, data was also gathered from novice users. 
After a quick overview of the operation of the pipette measurements were taken after the volunteers adjusted the pipette to 200, 50, 20 and 10 $\mu$L. 
After the first attempt was measured on the scale the volunteer was allowed to adjust the set point and re-measure until within 15\% of the target volume was 58 reached.

% For figure citations, please use "Fig." instead of "Figure".
%Nulla mi mi, Fig.~\ref{fig1} venenatis sed ipsum varius, volutpat euismod diam. Proin rutrum vel massa non gravida. Quisque tempor sem et dignissim rutrum. Lorem ipsum dolor sit amet, consectetur adipiscing elit. Morbi at justo vitae nulla elementum commodo eu id massa. In vitae diam ac augue semper tincidunt eu ut eros. Fusce fringilla erat porttitor lectus cursus, \nameref{S1_Video} vel sagittis arcu lobortis. Aliquam in enim semper, aliquam massa id, cursus neque. Praesent faucibus semper libero.

\begin{figure}[h]
\caption{{\bf Novice User efficacy.}
Novice users were asked to pipette volumes of 200, 50, 20 and 10 $\mu$L without the use of a scale.}
\label{fig3}
\end{figure}

%\begin{enumerate}
%\item{react}
%\item{diffuse free particles}
%\item{increment time by dt and go to 1}
%\end{enumerate}

% Results and Discussion can be combined.
\section*{Results and Discussion}

Our printed pipette had a comparable accuracy to a commercial pipette (Table~\ref{table1}).
The commercial pipette has a digital readout that allows the user to adjust precisely to set values where the printed pipette is adjusted by eye according to graduations.
This is certainly the cause of the lower accuracy and wider deviation although it's performance is impressive. 

\begin{table}[!ht]
\caption{
\bf{Comparison of Accuracy and Precision}}
\begin{tabular}{|c|c|c|c|}
\hline
    Target Volume & 20 $\mu$L & 50 $\mu$L & 200 $\mu$L  \\
    \hline
    Printed Pipette & 196.4 $\pm$ 2.2 & 53.5 $\pm$ 1.8 & 19.5 $\pm$ 0.6 \\
    Commercial Pipette & 204.5 $\pm$ 2.9 & 49.9 $\pm$ 0.1 & 19.9 $\pm$ 0.2 \\
    \hline
\end{tabular}
\begin{flushleft} Average measured volume by weight and standard deviation.
\end{flushleft}
\label{table1}
 \end{table}

The novice user data demonstrates that the pipette set point is user friendly and is capable of accurate measurements without validating with a scale (Figure~\ref{fig3}).


%\begin{table}[!ht]
%\begin{adjustwidth}{-2.25in}{0in} % Comment out/remove adjustwidth environment if table fits in text column.
%\caption{
%{\bf Table caption Nulla mi mi, venenatis sed ipsum varius, volutpat euismod diam.}}
%\begin{tabular}{|l|l|l|l|l|l|l|l|}
%\hline
%\multicolumn{4}{|l|}{\bf Heading1} & \multicolumn{4}{|l|}{\bf Heading2}\\ \hline
%$cell1 row1$ & cell2 row 1 & cell3 row 1 & cell4 row 1 & cell5 row 1 & cell6 row 1 & cell7 row 1 & cell8 row 1\\ \hline
%$cell1 row2$ & cell2 row 2 & cell3 row 2 & cell4 row 2 & cell5 row 2 & cell6 row 2 & cell7 row 2 & cell8 row 2\\ \hline
%$cell1 row3$ & cell2 row 3 & cell3 row 3 & cell4 row 3 & cell5 row 3 & cell6 row 3 & cell7 row 3 & cell8 row 3\\ \hline
%\end{tabular}
%\begin{flushleft} Table notes Phasellus venenatis, tortor nec vestibulum mattis, massa tortor interdum felis, nec pellentesque metus tortor nec nisl. Ut ornare mauris tellus, vel dapibus arcu suscipit sed.
%\end{flushleft}
%\label{table1}
%\end{adjustwidth}
%\end{table}

% Please do not create a heading level below \subsection. For 3rd level headings, use \paragraph{}. 

For more information, see \nameref{S1_Text}.

\section*{Supporting Information}

% Include only the SI item label in the subsection heading. Use the \nameref{label} command to cite SI items in the text.
%\subsection*{S1 Video}
%\label{S1_Video}
%{\bf Bold the first sentence.}  Maecenas convallis mauris sit amet sem ultrices gravida. Etiam eget sapien nibh. Sed ac ipsum eget enim egestas ullamcorper nec euismod ligula. Curabitur fringilla pulvinar lectus consectetur pellentesque.

%\subsection*{S1 Text}
%\label{S1_Text}
%{\bf Lorem Ipsum.} Maecenas convallis mauris sit amet sem ultrices gravida. Etiam eget sapien nibh. Sed ac ipsum eget enim egestas ullamcorper nec euismod ligula. Curabitur fringilla pulvinar lectus consectetur pellentesque.

%\subsection*{S1 Fig}
%\label{S1_Fig}
%{\bf Lorem Ipsum.} Maecenas convallis mauris sit amet sem ultrices gravida. Etiam eget sapien nibh. Sed ac ipsum eget enim egestas ullamcorper nec euismod ligula. Curabitur fringilla pulvinar lectus consectetur pellentesque.

%\subsection*{S2 Fig}
%\label{S2_Fig}
%{\bf Lorem Ipsum.} Maecenas convallis mauris sit amet sem ultrices gravida. Etiam eget sapien nibh. Sed ac ipsum eget enim egestas ullamcorper nec euismod ligula. Curabitur fringilla pulvinar lectus consectetur pellentesque.

%\subsection*{S1 Table}
%\label{S1_Table}
%{\bf Lorem Ipsum.} Maecenas convallis mauris sit amet sem ultrices gravida. Etiam eget sapien nibh. Sed ac ipsum eget enim egestas ullamcorper nec euismod ligula. Curabitur fringilla pulvinar lectus consectetur pellentesque.

%\section*{Acknowledgments}
%Cras egestas velit mauris, eu mollis turpis pellentesque sit amet. Interdum et malesuada fames ac ante ipsum primis in faucibus. Nam id pretium nisi. Sed ac quam id nisi malesuada congue. Sed interdum aliquet augue, at pellentesque quam rhoncus vitae.

\nolinenumbers

%\section*{References}
% Either type in your references using
% \begin{thebibliography}{}
% \bibitem{}
% Text
% \end{thebibliography}
%
% OR
%
% Compile your BiBTeX database using our plos2015.bst
% style file and paste the contents of your .bbl file
% here.
% 
\bibliographystyle{plos2015.bst}
\bibliography{references}



\end{document}

