% Template for PLoS
% Version 3.1 February 2015
%
% To compile to pdf, run:
% latex plos.template
% bibtex plos.template
% latex plos.template
% latex plos.template
% dvipdf plos.template
%
% % % % % % % % % % % % % % % % % % % % % %
%
% -- IMPORTANT NOTE
%
% This template contains comments intended 
% to minimize problems and delays during our production 
% process. Please follow the template instructions
% whenever possible.
%
% % % % % % % % % % % % % % % % % % % % % % % 
%
% Once your paper is accepted for publication, 
% PLEASE REMOVE ALL TRACKED CHANGES in this file and leave only
% the final text of your manuscript.
%
% There are no restrictions on package use within the LaTeX files except that 
% no packages listed in the template may be deleted.
%
% Please do not include colors or graphics in the text.
%
% Please do not create a heading level below \subsection. For 3rd level headings, use \paragraph{}.
%
% % % % % % % % % % % % % % % % % % % % % % %
%
% -- FIGURES AND TABLES
%
% Please include tables/figure captions directly after the paragraph where they are first cited in the text.
%
% DO NOT INCLUDE GRAPHICS IN YOUR MANUSCRIPT
% - Figures should be uploaded separately from your manuscript file. 
% - Figures generated using LaTeX should be extracted and removed from the PDF before submission. 
% - Figures containing multiple panels/subfigures must be combined into one image file before submission.
% For figure citations, please use "Fig." instead of "Figure".
% See http://www.plosone.org/static/figureGuidelines for PLOS figure guidelines.
%
% Tables should be cell-based and may not contain:
% - tabs/spacing/line breaks within cells to alter layout or alignment
% - vertically-merged cells (no tabular environments within tabular environments, do not use \multirow)
% - colors, shading, or graphic objects
% See http://www.plosone.org/static/figureGuidelines#tables for table guidelines.
%
% For tables that exceed the width of the text column, use the adjustwidth environment as illustrated in the example table in text below.
%
% % % % % % % % % % % % % % % % % % % % % % % %
%
% -- EQUATIONS, MATH SYMBOLS, SUBSCRIPTS, AND SUPERSCRIPTS
%
% IMPORTANT
% Below are a few tips to help format your equations and other special characters according to our specifications. For more tips to help reduce the possibility of formatting errors during conversion, please see our LaTeX guidelines at http://www.plosone.org/static/latexGuidelines
%
% Please be sure to include all portions of an equation in the math environment.
%
% Do not include text that is not math in the math environment. For example, CO2 will be CO\textsubscript{2}.
%
% Please add line breaks to long display equations when possible in order to fit size of the column. 
%
% For inline equations, please do not include punctuation (commas, etc) within the math environment unless this is part of the equation.
%
% % % % % % % % % % % % % % % % % % % % % % % % 
%
% Please contact latex@plos.org with any questions.
%
% % % % % % % % % % % % % % % % % % % % % % % %

\documentclass[10pt,letterpaper]{article}
\usepackage[top=0.85in,left=2.75in,footskip=0.75in]{geometry}

% Use adjustwidth environment to exceed column width (see example table in text)
\usepackage{changepage}

% Use Unicode characters when possible
\usepackage[utf8]{inputenc}

% textcomp package and marvosym package for additional characters
\usepackage{textcomp,marvosym}

% fixltx2e package for \textsubscript
\usepackage{fixltx2e}

% amsmath and amssymb packages, useful for mathematical formulas and symbols
\usepackage{amsmath,amssymb}

% cite package, to clean up citations in the main text. Do not remove.
\usepackage{cite}

% Use nameref to cite supporting information files (see Supporting Information section for more info)
\usepackage{nameref,hyperref}

% line numbers
\usepackage[right]{lineno}

% ligatures disabled
\usepackage{microtype}
\DisableLigatures[f]{encoding = *, family = * }

% rotating package for sideways tables
\usepackage{rotating}

\usepackage{multirow}

% Remove comment for double spacing
%\usepackage{setspace} 
%\doublespacing

% Text layout
\raggedright
\setlength{\parindent}{0.5cm}
\textwidth 5.25in 
\textheight 8.75in

% Bold the 'Figure #' in the caption and separate it from the title/caption with a period
% Captions will be left justified
\usepackage[aboveskip=1pt,labelfont=bf,labelsep=period,justification=raggedright,singlelinecheck=off]{caption}

% Use the PLoS provided BiBTeX style
\bibliographystyle{plos2015}

% Remove brackets from numbering in List of References
\makeatletter
\renewcommand{\@biblabel}[1]{\quad#1.}
\makeatother

% Leave date blank
\date{}

% Header and Footer with logo
\usepackage{lastpage,fancyhdr,graphicx}
\usepackage{epstopdf}
\pagestyle{myheadings}
\pagestyle{fancy}
\fancyhf{}
\lhead{\includegraphics[width=2.0in]{PLOS-submission.eps}}
\rfoot{\thepage/\pageref{LastPage}}
\renewcommand{\footrule}{\hrule height 2pt \vspace{2mm}}
\fancyheadoffset[L]{2.25in}
\fancyfootoffset[L]{2.25in}
\lfoot{\sf PLOS}

%% Include all macros below

\newcommand{\lorem}{{\bf LOREM}}
\newcommand{\ipsum}{{\bf IPSUM}}

%% END MACROS SECTION


\begin{document}
\vspace*{0.35in}

% Title must be 250 characters or less.
% Please capitalize all terms in the title except conjunctions, prepositions, and articles.
\begin{flushleft}
{\Large
\textbf\newline{3D-Printable Micropipette}
}
\newline
% Insert author names, affiliations and corresponding author email (do not include titles, positions, or degrees).
\\
Martin D. Brennan\textsuperscript{1},
Fahad F. Bokhari\textsuperscript{1},
David T. Eddington\textsuperscript{1,*},
%Name4 Surname\textsuperscript{2,\ddag},
%Name5 Surname\textsuperscript{2,\ddag},
%Name6 Surname\textsuperscript{2},
%Name7 Surname\textsuperscript{3,*},
%with the Lorem Ipsum Consortium\textsuperscript{\textpilcrow}
\\
\bigskip
\bf{1} Department of Bioengineering, University of Illinois at Chicago, Chicago, Illinois, USA
%\\
%\bf{2} Affiliation Dept/Program/Center, Institution Name, City, State, Country
%\\
%\bf{3} Affiliation Dept/Program/Center, Institution Name, City, State, Country
%\\
\bigskip

% Insert additional author notes using the symbols described below. Insert symbol callouts after author names as necessary.
% 
% Remove or comment out the author notes below if they aren't used.
%
% Primary Equal Contribution Note
%\Yinyang These authors contributed equally to this work.

% Additional Equal Contribution Note
% Also use this double-dagger symbol for special authorship notes, such as senior authorship.
%\ddag These authors also contributed equally to this work.

% Current address notes
%\textcurrency a Insert current address of first author with an address update
% \textcurrency b Insert current address of second author with an address update
% \textcurrency c Insert current address of third author with an address update

% Deceased author note
%\dag Deceased

% Group/Consortium Author Note
%\textpilcrow Membership list can be found in the Acknowledgments section.

% Use the asterisk to denote corresponding authorship and provide email address in note below.
* dte@uic.edu

\end{flushleft}
% Please keep the abstract below 300 words
\section*{Abstract}
Open source development of lab equipment is being facilitated, in part, by growing availability of 3D printing which allows the production of simple lab tools.
We developed 3D-printable parts that, along with a disposable syringe, form a micropipette.
Once assembled the pipette requires no calibration or validation with a scale.
Our printed micropipette is assessed in comparison to a commercial pipette demonstrating comparable performance in accuracy and precision and approaches ISO standard.

\linenumbers

\section*{Introduction}
The open source development model, initially applied to software, is thriving in the development of open source scientific equipment due in part to increasing access of 3D printing \cite{Baden2015,Pearce2014}.
Open design 3D-printable lab equipment is and attractive idea because, like open source software, it allows free access to technology that is otherwise inaccessible due to proprietary and/or financial barriers. 
Open design tools create opportunity for scientists and educational programs in remote or resource limited areas to participate with inexpensive and easy to make tools\cite{Baden2015}. 
Open source development also enables the development of custom solutions to meet unique applications not met by commercial products that are shared freely and are user modifiable\cite{Fullerton2014,Pearce2012}.
Some advanced, noteworthy open-source scientific equipment include a PCR device\cite{ChaiBiotechnologiesInc2015}, and a two-photon microscope\cite{Rosenegger2014} although even simple tools are impactful especially if the fabrication is simplified by 3D printing.
%The user is required to assemble the tools requiring them the learn how it functions.
%The user is then equipped to diagnose problems and enact their own fixes and changes which is limited when dealing with commercial products.

3D printing as a form of additive manufacturing has existed for decades although the recent availability of inexpensive desktop printers\cite{MakerbotIndustries2016,Reprap2015} have made it feasible for consumers to design and print prototypes and even functional parts, and consumer goods\cite{Fullerton2014}.
Proliferation of free CAD software[OpenSCAD, Blender, SketchUp] and design sharing sites\cite{MakerbotIndustries2016,NationalInstitutesofHealth2015,grabCAD,GitHubInc.2016} have also supported the growth and popularity of open designed parts and projects.
Some simple and clever printable parts that have emerged are ones that give a new function to a ubiquitous existing device such as drill bit attachment designed to hold centrifuge tubes, turning the drill into a centrifuge\cite{Garvey2009}.
Although this may make a rather crude centrifuge it may be and adequate method that only costs pennies compared to commercially available centrifuges.
Other examples of open-design research tools that utilize 3D printed parts include optics equipment\cite{Zhang2013}, microscopes\cite{Baden2014a,Walus2014}, syringe pumps\cite{Wijnen2014}, and reactionware\cite{Symes2012} to name a few.

There are a few open design printable micropipettes including a popular one which uses parts scavenged from a retractable pen in addition to the printed parts\cite{Baden2014}.
An air displacement pipette uses a piston operating principal to draw liquid into the pipette.
In a typical commercial pipette the piston is made gas-tight with a gasketed plunger inside a smooth barrel.
Consumer grade fused deposition modeling (FDM) printers are unable to form a smooth surface do to the formation of ridges that occur as each layer that is deposited.
The ridges formed by FDM make it impractical to form a gas-tight seal between moving parts, even with a gasket.
Printable open-design micropipettes get around this limitation by stretching a membrane over one end of a printed tube, which when pressed causes the displacement.
The displacement membrane can be made from any elastic material such as a latex glove.
After assembly this design requires the user to validate the volumes dispensed with a high precision scale.
%uncertianty  
Without verification with a scale the volumes dispensed can only be estimated based on theoretical displacement intended by the design.
There is no built-in feature such as a digital readout for the user to set the displacement to a desired volume.
The pipette can be calibrated to a specific volume with a set screw but if a new volume is desired it has to be re-adjusted and validated with a scale.
%In addition no open-design pipettes have be rigorously validated.

We present a new open design 3D-printable micropipette that works by actuating a 1 mL syringe to a user adjustable set-point.
This allows the user to set the pipette to a volume a priori by reading the graduations of the syringe barrel.
The scale marks on the syringe can be used to adjust the dispensed volume accurately without calibrating with a scale.
Our pipette also offers a simplified assembly requiring no glue, tape or permanent connections. 

%range of 226 ul, one tip size per pipette



%\begin{equation}\label{eq:schemeP} 
%D_{coll} = \frac{D_f+\frac{[S]^2}{K_D S_T} D_S} {1+\frac{[S]^2}{K_D S_T}}, 
%D_{sm} = \frac{D_f+ \frac{[S]}{K_D} D_S}{1+\frac{[S]}{K_D}},
%\end{equation}

% You may title this section "Methods" or "Models". 
% "Models" is not a valid title for PLoS ONE authors. However, PLoS ONE
% authors may use "Analysis" 
\section*{Materials and Methods}
\subsection*{Design}

Our printed pipette is designed to actuate a 1 mL syringe to a user set displacement.
The core of the design is two printed parts, the body and the plunger.
A 1 mL BD syringe twists to lock in the body part and is held into place by the syringe flanges.
The plunger part slides freely in the body part and actuates the syringe by pushing the thumb press. 
The pipette is spring loaded towards a set point which is adjustable by a set-screw. 
When the plunger is depressed the system locks when it reaches the ready point, where it is ready to draw in fluid.
The plunger is held in place with a notch and groove design which is then released with a lever drawing in fluid. 
The displacement is directly readable with the syringe graduations and is equal to the distance between the set and the ready point.
The pipette can also be pressed past the ready point to purge the transfered fluid completely from the pipette tip.   
An additional printed part adapts standard micropipette tips to the luer fitting of the pipette.
Additional materials required for assembly include a spring, a nut and bolt and washers.

Our printed pipette mimics commercial pipettes's design, function, and user operation, making it intuitive to use.
It also can reach to the bottom of a 15 mL conical tube.

%3D parts were designed in Blender and SolidWorks and printed by Fineline Prototyping via selective laser sintering. 


\begin{figure}
\caption{
{\bf CAD rendering of printable parts.}  Three parts are printed: the body, plunger shaft, and luer-lock adapter for pipette tips.
}
\label{fig1}
% Ideally: cad rendering showing the design and function of the pipette. 
% a) Un-assembled whole parts next to their cross-sections.  
% b) Assembly next to crossection
% c) close up of how the locking groove/mechanism works
% d) cross-section showing the two positions of the plunger, locked at the goove and stopped at the screw 
\end{figure}

\begin{figure}
\caption{
{\bf Photos of pipette.}  The pipette is composed of three printed parts, a 1 mL syringe and some additional hardware.  
}
\label{fig2}
% Ideally: Photo showing assembled pipette as well as parts
\end{figure}

\subsection*{Validation}
The pipette was validated and compared to a commercial pipette. 
The pipette was tested to see if the set point would drift between during use for example small, incremental movements of the screw or other parts causing the pipetted volume to drift. 
To test the pipette the same volume was pipetted 50 times without moving or re-adjusting the set point. 
The printed pipette’s accuracy and precision was characterized and compared to a commercial pipette. 
The printed pipette and a series of commercial pipettes were tested at volumes of 20 $\mu$L 50 $\mu$L and 200 $\mu$L. 
The printed pipette was adjusted to the the target volume by eye from the syringe graduations, and the resulting volume was measured and repeated 5 times to account for variability.
Measurements were taken from the commercial pipette in a similar way.

To test user friendliness and feasibility for using the pipette without verifying with a scale, data was also gathered from novice users. 
This group included researchers experienced with pipettes as well as undergraduates with little to no experience pipetting at all.
After a quick overview of the operation of the pipette measurements were taken after the volunteers adjusted the pipette to 200, 50, 20 and 10 $\mu$L. 

For the printed pipette the 1000 $\mu$L tips were used for the 200 $\mu$L measurement. 100 $\mu$L tips were used for the 50, 20 and 10 $\mu$L measurements.

compare the force required to press the button.

compare the weight of the pipettes

%After the first attempt was measured on the scale the volunteer was allowed to adjust the set point and re-measure until within 15\% of the target volume was reached.

% For figure citations, please use "Fig." instead of "Figure".
%Nulla mi mi, Fig.~\ref{fig1} venenatis sed ipsum varius, volutpat euismod diam. Proin rutrum vel massa non gravida. Quisque tempor sem et dignissim rutrum. Lorem ipsum dolor sit amet, consectetur adipiscing elit. Morbi at justo vitae nulla elementum commodo eu id massa. In vitae diam ac augue semper tincidunt eu ut eros. Fusce fringilla erat porttitor lectus cursus, \nameref{S1_Video} vel sagittis arcu lobortis. Aliquam in enim semper, aliquam massa id, cursus neque. Praesent faucibus semper libero.

\begin{figure}[h]
\caption{{\bf Novice User efficacy.}
Novice users were asked to pipette volumes of 200, 50, 20 and 10 $\mu$L without the use of a scale.}
\label{fig3}
\end{figure}

%\begin{enumerate}
%\item{react}
%\item{diffuse free particles}
%\item{increment time by dt and go to 1}
%\end{enumerate}

% Results and Discussion can be combined.
\section*{Results and Discussion}

Our printed pipette had a comparable accuracy to a commercial pipette (Table~\ref{table1}).
The commercial pipette has a digital readout that allows the user to adjust precisely to set values where the printed pipette is adjusted by eye according to graduations.
Due to being unable to re-set the printed pipette back to the exact mechanical location it may be hard to compare directly with the commercial pipettes.
This is certainly the cause of the lower accuracy and wider deviation although it's performance is impressive.
Despite this additional variability the printed pipette's accuracy and precision approaches the ISO 8655 standard.

The Novice user data demonstrates that even someone inexperienced with pipetting and unfamiliar with our design is able to use out pipette accurately with out the use of a scale. 
The systematic error trended towards over estimating the displacement. 
This is likely due the function of adjusting the volume where it is easier to start at a lower displacement and increase it by unscrewing, or releasing compression of the spring.
This explains the negative systematic error as the user avoids 'going over'.

This pipette improves on existing open design pipettes in several ways. 
It requires only two printed parts, a syringe, and some hardware.
It is able to reach into a 15 mL conical tube which is a routine requirement for cell culture.
No permanent connections using tape or glue are required for assembly allowing worn out or broken part to be replaced easily.
Assembly does not require a membrane that may wear or stretch and require tedious replacement.
No validation or calibration is required.
The pipette can be assembled and used to accurately pipette micro volumes without first validating with a scale.

The only major limitation of this design compared to the briopipette is the option for a pipette tip ejector, although incorporation of a similar design in our pipette would not leave room for it to reach into a 15 mL conical tube. 
Our pipette requires a syringe and some additional hardware where briopipette's additional parts can be obtained virtually anywhere.
The briopipette also allows step-wise shifts in displacement allowing the user to quickly change to a relative volume


% Please add the following required packages to your document preamble:
% \usepackage{multirow}
\begin{table}[]
\centering
\caption{
\bf{Comparison of error between the printed and commercially produced pipette}}
\begin{tabular}{llllll}
                  Volume &   & Systematic error &  & Random error & \\
                  \hline
\multirow{2}{*}{10 $\mu$L} & printed & 2.17 \% & $\pm$0.217$\mu$L  & 1.96\% & $\pm$0.196$\mu$L   \\
                  & commercial & 3\% & $\pm$0.3$\mu$L  & 1\% & $\pm$0.1$\mu$L  \\
                  \hline
\multirow{2}{*}{20 $\mu$L} & printed  & 0.33\% & $\pm$0.066$\mu$L & 1.18 \% & $\pm$0.236$\mu$L \\
                  & commercial & 2.50\% & $\pm$0.5$\mu$L & 0.70\% & $\pm$0.14$\mu$L \\
                  \hline
\multirow{2}{*}{50 $\mu$L} & printed & 1.17\% & $\pm$0.585$\mu$L & 0.50\% & $\pm$0.25$\mu$L \\
                  & commercial & 1\% & $\pm$0.5$\mu$L & 0.30\% & $\pm$0.15$\mu$L \\
                  \hline
\multirow{2}{*}{200 $\mu$L} & printed & 2.52\% & $\pm$5.04$\mu$L & 0.69\% & $\pm$1.38$\mu$L  \\
                  & commercial & 0.6\% & $\pm$1.2$\mu$L & 0.20\% & $\pm$0.4$\mu$L
\end{tabular}
\begin{flushleft} Error assessed for average measured volume by weight of water compared with published specifications of commercial pipette. [Alos compare to data taken with commercial pipettes in our lab]
\end{flushleft}
\label{table0}
\end{table}

\begin{table}[]
\centering
\caption{\bfseries{Commercial and Printed pipettes compared to ISO standard 8655.}}
\label{my-label}
\begin{tabular}{lllllll}
Volume & Pipette & Mean   & SE    & \% SE & RE & \% RE \\
       & ISO, 20-200 $\mu$L & 200 $\mu$L       & $\pm$1.60         & $\pm$0.08          & 0.60 & 0.30 \\
200 $\mu$L & Commercial & 198.23       & -1.77        & -0.88         & 1.14 & 0.58 \\
       & Printed  & 193.40       & -6.60        & -3.30         & 2.86 & 1.48 \\
       &                     &              &              &               &      &      \\
       & ISO, 20-200 $\mu$L & 50 $\mu$L        & $\pm$1.60         &$\pm$3.20          & 0.60 & 1.20 \\
50 $\mu$L  & Commercial   & 48.89        & -1.11        & -2.23         & 0.71 & 1.46 \\
       & Printed      & 49.62        & -0.38        & -0.76         & 1.26 & 2.53 \\
       &                     &              &              &               &      &      \\
       & ISO, 20-200 $\mu$L & 20 $\mu$L        & $\pm$1.60         & $\pm$8.00          & 0.60 & 3.00 \\
20 $\mu$L  & Commercial   & 20.16        & 0.16         & 0.81          & 0.16 & 0.81 \\
       & Printed     & 18.70        & -1.30        & -6.48         & 0.38 & 2.01 \\
       &                     &              &              &               &      &      \\
       & ISO, 10-100 $\mu$L & 10 $\mu$L        & $\pm$0.80         & $\pm$8.00          & 0.30 & 3.00 \\
10 $\mu$L  & Commercial   & 9.88         & -0.12        & -1.23         & 0.34 & 3.39 \\
       & Printed      & 11.95        & 1.95         & 19.52         & 0.73 & 6.08 \\
       &                     &              &              &               &      &      \\
       & ISO, 20-200 $\mu$L & 10 $\mu$L        & $\pm$1.60         & $\pm$16.00         & 0.60 & 6.00 \\
10 $\mu$L* & Commercial   & 9.88         & -0.12        & -1.23         & 0.34 & 3.39 \\
       & Printed      & 11.95        & 1.95         & 19.52         & 0.73 & 6.08
\end{tabular}
\begin{flushleft} Commercial and Printed pipettes compared to ISO standard 8655.
\end{flushleft}
\label{table1}
\end{table}

The novice user data demonstrates that the pipette set point is user friendly and is capable of accurate measurements without validating with a scale (Figure~\ref{fig3}).

\begin{table}[]
\centering9
\caption{
\bf{Novice Users' first attempt}}
\begin{tabular}{llll}
Target volume & Average $\pm$ SD & Systematic error  & Random error\\
\hline
10 $\mu$L         & 7.6 $\pm$ 3.3     & -23.80\% & 9.95\% \\
\hline
20 $\mu$L         & 18.1 $\pm$ 4.4   & -9.60\%   & 6.66\% \\
\hline
50 $\mu$L         & 45.3 $\pm$ 4.6   & -9.48\%   & 2.75\% \\
\hline
200 $\mu$L       & 185.2 $\pm$ 5.6 & -7.43\%   & 0.84\%
\end{tabular}
\begin{flushleft} Error assessed for average measured volume by weight of water compared with published specifications of commercial pipette.
\end{flushleft}
\label{table2}
\end{table}

%\begin{table}[!ht]
%\begin{adjustwidth}{-2.25in}{0in} % Comment out/remove adjustwidth environment if table fits in text column.
%\caption{
%{\bf Table caption Nulla mi mi, venenatis sed ipsum varius, volutpat euismod diam.}}
%\begin{tabular}{|l|l|l|l|l|l|l|l|}
%\hline
%\multicolumn{4}{|l|}{\bf Heading1} & \multicolumn{4}{|l|}{\bf Heading2}\\ \hline
%$cell1 row1$ & cell2 row 1 & cell3 row 1 & cell4 row 1 & cell5 row 1 & cell6 row 1 & cell7 row 1 & cell8 row 1\\ \hline
%$cell1 row2$ & cell2 row 2 & cell3 row 2 & cell4 row 2 & cell5 row 2 & cell6 row 2 & cell7 row 2 & cell8 row 2\\ \hline
%$cell1 row3$ & cell2 row 3 & cell3 row 3 & cell4 row 3 & cell5 row 3 & cell6 row 3 & cell7 row 3 & cell8 row 3\\ \hline
%\end{tabular}
%\begin{flushleft} Table notes Phasellus venenatis, tortor nec vestibulum mattis, massa tortor interdum felis, nec pellentesque metus tortor nec nisl. Ut ornare mauris tellus, vel dapibus arcu suscipit sed.
%\end{flushleft}
%\label{table1}
%\end{adjustwidth}
%\end{table}

% Please do not create a heading level below \subsection. For 3rd level headings, use \paragraph{}. 

For more information, see \nameref{S1_Text}.

\section*{Supporting Information}

% Include only the SI item label in the subsection heading. Use the \nameref{label} command to cite SI items in the text.
%\subsection*{S1 Video}
%\label{S1_Video}
%{\bf Bold the first sentence.}  Maecenas convallis mauris sit amet sem ultrices gravida. Etiam eget sapien nibh. Sed ac ipsum eget enim egestas ullamcorper nec euismod ligula. Curabitur fringilla pulvinar lectus consectetur pellentesque.

%\subsection*{S1 Text}
%\label{S1_Text}
%{\bf Lorem Ipsum.} Maecenas convallis mauris sit amet sem ultrices gravida. Etiam eget sapien nibh. Sed ac ipsum eget enim egestas ullamcorper nec euismod ligula. Curabitur fringilla pulvinar lectus consectetur pellentesque.

%\subsection*{S1 Fig}
%\label{S1_Fig}
%{\bf Lorem Ipsum.} Maecenas convallis mauris sit amet sem ultrices gravida. Etiam eget sapien nibh. Sed ac ipsum eget enim egestas ullamcorper nec euismod ligula. Curabitur fringilla pulvinar lectus consectetur pellentesque.

%\subsection*{S2 Fig}
%\label{S2_Fig}
%{\bf Lorem Ipsum.} Maecenas convallis mauris sit amet sem ultrices gravida. Etiam eget sapien nibh. Sed ac ipsum eget enim egestas ullamcorper nec euismod ligula. Curabitur fringilla pulvinar lectus consectetur pellentesque.

%\subsection*{S1 Table}
%\label{S1_Table}
%{\bf Lorem Ipsum.} Maecenas convallis mauris sit amet sem ultrices gravida. Etiam eget sapien nibh. Sed ac ipsum eget enim egestas ullamcorper nec euismod ligula. Curabitur fringilla pulvinar lectus consectetur pellentesque.

%\section*{Acknowledgments}
%Cras egestas velit mauris, eu mollis turpis pellentesque sit amet. Interdum et malesuada fames ac ante ipsum primis in faucibus. Nam id pretium nisi. Sed ac quam id nisi malesuada congue. Sed interdum aliquet augue, at pellentesque quam rhoncus vitae.

\nolinenumbers

%\section*{References}
% Either type in your references using
% \begin{thebibliography}{}
% \bibitem{}
% Text
% \end{thebibliography}
%
% OR
%
% Compile your BiBTeX database using our plos2015.bst
% style file and paste the contents of your .bbl file
% here.
% 
\bibliographystyle{plos2015.bst}
\bibliography{references}



\end{document}

